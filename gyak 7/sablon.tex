\documentclass{article}
\usepackage[english,magyar]{babel}
\usepackage{t1enc}
\usepackage{amsmath}
\usepackage{amsfonts}
\usepackage{mathtools}
\usepackage{hulipsum}
\usepackage{xcolor}
\usepackage{colortbl}
\usepackage{ifthen}
\usepackage{array}
\usepackage{pgffor}
\usepackage{multicol}
\usepackage{enumitem}


\begin{document}

\begin{enumerate}[label=(\alph*)]
	\item Defniáljuk az arg max operátort, és teszteljük a viselkedését kiemelt matematikai módban:
	\[
		\underset{x\in[0, 1]}{x^*:= \text{arg max } x \text{log}_2(x)}.
	\]
	
	\item Defniáljuk és teszteljük a felső egészrész függvényt:
	\[
			\lceil x\rceil = \bigg\lceil \frac{5}{3} \bigg\rceil
	\]
	
	\item Defniáljuk a várható értéket és feltételes várható értéket (mint egy- ill. kétargumentumos parancsokat automatikusan méretezett zárójellel) és gépeljük
be velük a következő formulát!

	\[
		\mathbb{E} \Bigg[\sum_{i=1}^N X_i \Bigg] = \mathbb{E} \Bigg[ \mathbb{E} \Bigg[ \sum_{i=1}^N X_i \bigg\vert N \Bigg]\Bigg]
	\]

\end{enumerate}


Második Feladat

\newenvironment{keynote}[1]{\vspace{1ex}\hrule\vspace{1ex
}\begin{center}\textbf{#1}\end{center}}{\vspace{1ex}\hrule\vspace{1ex}}

\newcounter{keynotes}
\newcounter{sub-keynotes}[section]

\begin{keynote}{cím}
	\hulipsum[2]
	
	\newcommand{\kgitem}{\noindent\stepcounter{keynotes}\fbox{\makebox[0.5cm]{\thekeynotes}}}
	\kgitem
	
\end{keynote}

Negyedik feladat

\newcolumntype{A}{>{\ifthenelse{\isodd{\value{rownum}}}{\color{gray}}{\color{white}}}c}

\rowcolors{1}{white}{gray}
{\begin{tabular}{AAAA}
	1 & 2 & 3 & 4 \\
	1 & 2 & 3 & 4 \\
	1 & 2 & 3 & 4 \\
	1 & 2 & 3 & 4 \\
	1 & 2 & 3 & 4 \\
	1 & 2 & 3 & 4 \\
\end{tabular}


\end{document}