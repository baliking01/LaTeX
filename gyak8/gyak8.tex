\documentclass[aspectratio=169,12pt]{beamer}
\usepackage[english,magyar]{babel}
\usepackage{t1enc}
\usepackage{xcolor}
\usepackage{hyperref}
\usepackage{amsthm}
\usepackage{graphicx}
\usepackage{lipsum}
\usepackage{hulipsum}
\usepackage{blindtext}

\usetheme{Copenhagen}
\usecolortheme{wolverine}
\useoutertheme{smoothbars}
\useinnertheme{circles}

\author{Nagy Balázs}
\title{\LaTeX\ gyakorlat}
\subtitle{Egy remek alkotás}
\institute{Miskolci Egyetem Gépészmérnöki és Informatikai kara}

\begin{document}
\section{Bevezetés}
\subsection{Címek}
\begin{frame}[plain]
	\maketitle
\end{frame}

\begin{frame}
	\sectionpage
\end{frame}
\begin{frame}[plain]{Tartalomjegyzék}
	\tableofcontents[currentsection]
\end{frame}

\begin{frame}{Valami cím}{Egy kevésbé érdekes alcím}
	\hulipsum[1]
\end{frame}

\begin{frame}[allowframebreaks]{Cím 2}{Még egy hasonló alcím}
	\hulipsum
\end{frame}

\subsection{Verbatim elhelyezése}
\begin{frame}[fragile]{Valóban érdekes cím}{alatt}
	\hulipsum[1]
	\begin{verbatim}
		Ez egy verbatim környezet
	\end{verbatim}
\end{frame}

\subsection{Fénykép elhelyezése}
\begin{frame}
	\begin{columns}[c]
		\begin{column}{0.5\linewidth}
			\begin{itemize}
				\item első
				\item második
			\end{itemize}
			
			\begin{enumerate}
				\item harmadik
				\item negyedik
			\end{enumerate}
		\end{column}
		
		\begin{column}{0.5\linewidth}
			\begin{figure}
				\centering
				\includegraphics[scale=0.1]{irén.png}
				\caption{Egy 21. századi magyar állampolgár}
			\end{figure}		
		\end{column}
	\end{columns}
\end{frame}

\subsection{Block-ok}
\begin{frame}
	\begin{block}{Egy szép doboz}
		A block tartalma
	\end{block}

	\pause	
	
	\begin{exampleblock}{Példák számára}
		Néhány hasznos példa
	\end{exampleblock}
	
	\begin{alertblock}{}
		Magyarország előre megy, nem hátra!
	\end{alertblock}
\end{frame}

\section{Elmélet}
\subsection{Elhelyezés}
\begin{frame}
	\sectionpage
\end{frame}
\begin{frame}[plain]{Tartalomjegyzék II}
	\tableofcontents[currentsection]
\end{frame}

\begin{frame}<1, 2>[allowframebreaks]
	\begin{theorem}
		\hulipsum[2]
	\end{theorem}
	
		
	\begin{proof}
		\hulipsum[5]
	\end{proof}
\end{frame}


\subsection{Semi-verbatim}
\begin{semiverbatim}
	\\begin\{\alert{enumerate}\}
	\\item 1
	\\item 2
	\\item ...
		\color{green}\\begin\{\alert{enumerate}\}
			\\item a
			\\item b
			\\item c
		\\end\{enumerate\}
	\color{black}\\end\{enumerate\}
\end{semiverbatim}

\begin{frame}
	\begin{itemize}
		\item minden slide-on
		\item<1> egy item
		\item<2-> csak a második dia után
		\item\alert<3>{csak a 3. slide-on más színű}
	\end{itemize}
\end{frame}

\begin{frame}
	\onslide<2>{Valami, 2. slide}
	\only<1>{csak az első dián}
	\visible<3>{Látható szöveg a 3. dián}
\end{frame}

\begin{frame}
	\begin{itemize}
		\item Egy csodálatos fénykép
		\item<1> 
		\begin{figure}
				\includegraphics[scale=0.3]{blob.jpg}
		\end{figure}				
		\item<2>
		\begin{figure}
				\includegraphics[scale=0.3]{blob.jpg}
		\end{figure}	
	\end{itemize}
\end{frame}

\end{document}