\documentclass[12pt, twoside, twocolumn]{article}
\usepackage[english,magyar]{babel}
\usepackage{t1enc}
\usepackage{geometry}
\usepackage{fancyhdr}
\usepackage{hyperref}
\usepackage{xcolor}
\usepackage{hulipsum}

\author{Nagy Balázs}
\title{2. gyakorlat}
\renewcommand{\thefootnote}{\fnsymbol{footnote}}
\geometry{margin=3cm, outer=5cm, bindingoffset=1cm,
marginparwidth=3cm, marginparsep=0.5cm, columnsep=2cm}
%\pagestyle{headings}
\pagestyle{myheadings}
\markboth{N. Balázs}{gyak2}

\begin{document}
\setcounter{secnumdepth}{5}
\setcounter{tocdepth}{5}
\maketitle

\begin{abstract}
\hulipsum[4]
\footnote[5]{lábjegyzet}
\end{abstract}
\clearpage

\pagenumbering{roman}
\tableofcontents
\pagenumbering{arabic}

\clearpage

\section[Prologue]{Bevezetés\footnote[4]{A bekezdés címe}}

\subsection{Első szekció}
\hulipsum[4]

\subsection{Második szekció}
\marginpar{\hulipsum[1]}
\hulipsum[13]

\section{Valami kellően hosszú cím}
\subsection{Első szint}
\subsubsection{Második szint}
\paragraph{Harmadik szint}
\subparagraph{Negyedik szint}

\begin{quote}
Ez egy nagyon bölcs idézet.
\end{quote}

\begin{quotation}
A MÁV idén 69 millió forint veszteséget szenvedett
\end{quotation}

\begin{verse}
\hulipsum[2-3]
\end{verse}

\appendix
\section{Domestic problems}
\subsection{A roma kisebbség életéról}
\hulipsum[15]
\subsection{Miért igénylik a CSOK-ot?}
\hulipsum[12]

\section{A megoldás}
\subsection{A cigány telepek felszámolása}
\hulipsum[8]
\subsection{Konszolidációs lépések}
\marginpar{\hulipsum[1]}
\hulipsum[9]
\end{document}