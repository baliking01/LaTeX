\documentclass{article}
\usepackage[english,magyar]{babel}
\usepackage{t1enc}
\usepackage{amsthm}
\usepackage{listings}
\usepackage{algorithm}
\usepackage{hulipsum}
\usepackage{float}
\usepackage{newfloat}
\usepackage{caption}
\usepackage{algpseudocode}

\theoremstyle{plain}
\newtheorem*{tet1}{Tétel}
\newtheorem{tet2}{Tétel}
\newtheorem{lemma}[tet2]{Lemma}

\theoremstyle{definition}
\newtheorem*{defin1}{Definíció}
\newtheorem{defin2}{Definíció}

\theoremstyle{remark}
\newtheorem*{megjegyzés}{Megjegyzés}
\newtheorem{feladat}{Feladat}[section]

\floatstyle{ruled}
\newfloat{forraskod}{thp}{los}
\floatname{forraskod}{Forraskod}

\floatname{lstlisting}{Algoritmus}



\begin{document}
	\listof{forraskod}{Forráskódok listája}
	\listof{lstlisting}{Algoritmusok}
	\pagebreak
	
	\begin{defin1}
	\hulipsum[1]
	\end{defin1}
	
	\begin{tet1}[Pitagorasz]
	\hulipsum[1]
	\end{tet1}ű
	
	\begin{proof}
	\hulipsum[1]
	\end{proof}
	
\pagebreak	
	
	\begin{lemma}
	\hulipsum[1]
	\end{lemma}
	
	\begin{defin2}
	\hulipsum[1]
	\end{defin2}
	
	\begin{tet2}[Riemann-integrál]
	\hulipsum[3]
	\end{tet2}
	
\pagebreak

	\section{Section}
		\begin{feladat}
		\hulipsum[1]
		\end{feladat}
		
		\begin{feladat}
		\hulipsum[1]
		\end{feladat}
		
		\begin{feladat}
		\hulipsum[1]
		\end{feladat}
	
	\section{Theorems}
		\begin{tet2}[Some famous theorem]
		\hulipsum[1]
		\end{tet2}
		
		\begin{proof}
		\hulipsum[1]
		\end{proof}
		
		\begin{megjegyzés}
		\hulipsum[1]
		\end{megjegyzés}
		
Nem végrehajtott parancs \verb|\LaTeX|\par
\verb|\section{•}|

\begin{forraskod}\caption {Tétel generálása}
	\begin{verbatim}
	 \begin{tet1}[Újabb tétel]
	 \hulipsum[0]
	 \end{tet1}\end{verbatim}
\end{forraskod}

\hulipsum

\begin{forraskod}\caption{Lista}
	\begin{verbatim}
	\begin{itemize}
    	\item One
    	\item Two
    	\item Three
	\end{itemize}
	\end{verbatim}
\end{forraskod}


\begin{lstlisting}[language=c, numbers=left, tabsize=5, frame=single, caption={Fast Inverse Square Root (Quake III Algorithm)}]
  float Q_rsqrt( float number ){
  	long i;
  	float x2, y;
  	const float threehalfs = 1.5F;

  	x2 = number * 0.5F;
  	y  = number;
  	i  = * ( long * ) &y;
  	i  = 0x5f3759df - ( i >> 1 );
  	y  = * ( float * ) &i;
  	y  = y * ( threehalfs - ( x2 * y * y ) );
  	
  	return y;
  
}
\end{lstlisting}
\pagebreak

\begin{algorithmic}[1]
	\Procedure{Quicksort}{@A, a, b}
		\Require A írható tömb
		\Require 1 $\leq$ a $\leq$ b $\leq$ Hossz|A| indexek
		\Ensure a-b indextartományt rendezzük
		\If{a = b}
		\State \Return A
		\Else
		\State \Call{Feloszt}{@A, a, b, A(a), @q}
		\State \Call{Quicksort}{@A, a, q}
		\State \Call{Quicksort}{@A, q+1, b}
		\State \Return A
		\EndIf
		
	\EndProcedure
\end{algorithmic}

\end{document}